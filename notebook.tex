
% Default to the notebook output style

    


% Inherit from the specified cell style.




    
\documentclass[11pt]{article}

    
    
    \usepackage[T1]{fontenc}
    % Nicer default font (+ math font) than Computer Modern for most use cases
    \usepackage{mathpazo}

    % Basic figure setup, for now with no caption control since it's done
    % automatically by Pandoc (which extracts ![](path) syntax from Markdown).
    \usepackage{graphicx}
    % We will generate all images so they have a width \maxwidth. This means
    % that they will get their normal width if they fit onto the page, but
    % are scaled down if they would overflow the margins.
    \makeatletter
    \def\maxwidth{\ifdim\Gin@nat@width>\linewidth\linewidth
    \else\Gin@nat@width\fi}
    \makeatother
    \let\Oldincludegraphics\includegraphics
    % Set max figure width to be 80% of text width, for now hardcoded.
    \renewcommand{\includegraphics}[1]{\Oldincludegraphics[width=.8\maxwidth]{#1}}
    % Ensure that by default, figures have no caption (until we provide a
    % proper Figure object with a Caption API and a way to capture that
    % in the conversion process - todo).
    \usepackage{caption}
    \DeclareCaptionLabelFormat{nolabel}{}
    \captionsetup{labelformat=nolabel}

    \usepackage{adjustbox} % Used to constrain images to a maximum size 
    \usepackage{xcolor} % Allow colors to be defined
    \usepackage{enumerate} % Needed for markdown enumerations to work
    \usepackage{geometry} % Used to adjust the document margins
    \usepackage{amsmath} % Equations
    \usepackage{amssymb} % Equations
    \usepackage{textcomp} % defines textquotesingle
    % Hack from http://tex.stackexchange.com/a/47451/13684:
    \AtBeginDocument{%
        \def\PYZsq{\textquotesingle}% Upright quotes in Pygmentized code
    }
    \usepackage{upquote} % Upright quotes for verbatim code
    \usepackage{eurosym} % defines \euro
    \usepackage[mathletters]{ucs} % Extended unicode (utf-8) support
    \usepackage[utf8x]{inputenc} % Allow utf-8 characters in the tex document
    \usepackage{fancyvrb} % verbatim replacement that allows latex
    \usepackage{grffile} % extends the file name processing of package graphics 
                         % to support a larger range 
    % The hyperref package gives us a pdf with properly built
    % internal navigation ('pdf bookmarks' for the table of contents,
    % internal cross-reference links, web links for URLs, etc.)
    \usepackage{hyperref}
    \usepackage{longtable} % longtable support required by pandoc >1.10
    \usepackage{booktabs}  % table support for pandoc > 1.12.2
    \usepackage[inline]{enumitem} % IRkernel/repr support (it uses the enumerate* environment)
    \usepackage[normalem]{ulem} % ulem is needed to support strikethroughs (\sout)
                                % normalem makes italics be italics, not underlines
    

    
    
    % Colors for the hyperref package
    \definecolor{urlcolor}{rgb}{0,.145,.698}
    \definecolor{linkcolor}{rgb}{.71,0.21,0.01}
    \definecolor{citecolor}{rgb}{.12,.54,.11}

    % ANSI colors
    \definecolor{ansi-black}{HTML}{3E424D}
    \definecolor{ansi-black-intense}{HTML}{282C36}
    \definecolor{ansi-red}{HTML}{E75C58}
    \definecolor{ansi-red-intense}{HTML}{B22B31}
    \definecolor{ansi-green}{HTML}{00A250}
    \definecolor{ansi-green-intense}{HTML}{007427}
    \definecolor{ansi-yellow}{HTML}{DDB62B}
    \definecolor{ansi-yellow-intense}{HTML}{B27D12}
    \definecolor{ansi-blue}{HTML}{208FFB}
    \definecolor{ansi-blue-intense}{HTML}{0065CA}
    \definecolor{ansi-magenta}{HTML}{D160C4}
    \definecolor{ansi-magenta-intense}{HTML}{A03196}
    \definecolor{ansi-cyan}{HTML}{60C6C8}
    \definecolor{ansi-cyan-intense}{HTML}{258F8F}
    \definecolor{ansi-white}{HTML}{C5C1B4}
    \definecolor{ansi-white-intense}{HTML}{A1A6B2}

    % commands and environments needed by pandoc snippets
    % extracted from the output of `pandoc -s`
    \providecommand{\tightlist}{%
      \setlength{\itemsep}{0pt}\setlength{\parskip}{0pt}}
    \DefineVerbatimEnvironment{Highlighting}{Verbatim}{commandchars=\\\{\}}
    % Add ',fontsize=\small' for more characters per line
    \newenvironment{Shaded}{}{}
    \newcommand{\KeywordTok}[1]{\textcolor[rgb]{0.00,0.44,0.13}{\textbf{{#1}}}}
    \newcommand{\DataTypeTok}[1]{\textcolor[rgb]{0.56,0.13,0.00}{{#1}}}
    \newcommand{\DecValTok}[1]{\textcolor[rgb]{0.25,0.63,0.44}{{#1}}}
    \newcommand{\BaseNTok}[1]{\textcolor[rgb]{0.25,0.63,0.44}{{#1}}}
    \newcommand{\FloatTok}[1]{\textcolor[rgb]{0.25,0.63,0.44}{{#1}}}
    \newcommand{\CharTok}[1]{\textcolor[rgb]{0.25,0.44,0.63}{{#1}}}
    \newcommand{\StringTok}[1]{\textcolor[rgb]{0.25,0.44,0.63}{{#1}}}
    \newcommand{\CommentTok}[1]{\textcolor[rgb]{0.38,0.63,0.69}{\textit{{#1}}}}
    \newcommand{\OtherTok}[1]{\textcolor[rgb]{0.00,0.44,0.13}{{#1}}}
    \newcommand{\AlertTok}[1]{\textcolor[rgb]{1.00,0.00,0.00}{\textbf{{#1}}}}
    \newcommand{\FunctionTok}[1]{\textcolor[rgb]{0.02,0.16,0.49}{{#1}}}
    \newcommand{\RegionMarkerTok}[1]{{#1}}
    \newcommand{\ErrorTok}[1]{\textcolor[rgb]{1.00,0.00,0.00}{\textbf{{#1}}}}
    \newcommand{\NormalTok}[1]{{#1}}
    
    % Additional commands for more recent versions of Pandoc
    \newcommand{\ConstantTok}[1]{\textcolor[rgb]{0.53,0.00,0.00}{{#1}}}
    \newcommand{\SpecialCharTok}[1]{\textcolor[rgb]{0.25,0.44,0.63}{{#1}}}
    \newcommand{\VerbatimStringTok}[1]{\textcolor[rgb]{0.25,0.44,0.63}{{#1}}}
    \newcommand{\SpecialStringTok}[1]{\textcolor[rgb]{0.73,0.40,0.53}{{#1}}}
    \newcommand{\ImportTok}[1]{{#1}}
    \newcommand{\DocumentationTok}[1]{\textcolor[rgb]{0.73,0.13,0.13}{\textit{{#1}}}}
    \newcommand{\AnnotationTok}[1]{\textcolor[rgb]{0.38,0.63,0.69}{\textbf{\textit{{#1}}}}}
    \newcommand{\CommentVarTok}[1]{\textcolor[rgb]{0.38,0.63,0.69}{\textbf{\textit{{#1}}}}}
    \newcommand{\VariableTok}[1]{\textcolor[rgb]{0.10,0.09,0.49}{{#1}}}
    \newcommand{\ControlFlowTok}[1]{\textcolor[rgb]{0.00,0.44,0.13}{\textbf{{#1}}}}
    \newcommand{\OperatorTok}[1]{\textcolor[rgb]{0.40,0.40,0.40}{{#1}}}
    \newcommand{\BuiltInTok}[1]{{#1}}
    \newcommand{\ExtensionTok}[1]{{#1}}
    \newcommand{\PreprocessorTok}[1]{\textcolor[rgb]{0.74,0.48,0.00}{{#1}}}
    \newcommand{\AttributeTok}[1]{\textcolor[rgb]{0.49,0.56,0.16}{{#1}}}
    \newcommand{\InformationTok}[1]{\textcolor[rgb]{0.38,0.63,0.69}{\textbf{\textit{{#1}}}}}
    \newcommand{\WarningTok}[1]{\textcolor[rgb]{0.38,0.63,0.69}{\textbf{\textit{{#1}}}}}
    
    
    % Define a nice break command that doesn't care if a line doesn't already
    % exist.
    \def\br{\hspace*{\fill} \\* }
    % Math Jax compatability definitions
    \def\gt{>}
    \def\lt{<}
    % Document parameters
    \title{blog\_post\_making\_using\_bs\_ds\_no\_EDA}
    
    
    

    % Pygments definitions
    
\makeatletter
\def\PY@reset{\let\PY@it=\relax \let\PY@bf=\relax%
    \let\PY@ul=\relax \let\PY@tc=\relax%
    \let\PY@bc=\relax \let\PY@ff=\relax}
\def\PY@tok#1{\csname PY@tok@#1\endcsname}
\def\PY@toks#1+{\ifx\relax#1\empty\else%
    \PY@tok{#1}\expandafter\PY@toks\fi}
\def\PY@do#1{\PY@bc{\PY@tc{\PY@ul{%
    \PY@it{\PY@bf{\PY@ff{#1}}}}}}}
\def\PY#1#2{\PY@reset\PY@toks#1+\relax+\PY@do{#2}}

\expandafter\def\csname PY@tok@w\endcsname{\def\PY@tc##1{\textcolor[rgb]{0.73,0.73,0.73}{##1}}}
\expandafter\def\csname PY@tok@c\endcsname{\let\PY@it=\textit\def\PY@tc##1{\textcolor[rgb]{0.25,0.50,0.50}{##1}}}
\expandafter\def\csname PY@tok@cp\endcsname{\def\PY@tc##1{\textcolor[rgb]{0.74,0.48,0.00}{##1}}}
\expandafter\def\csname PY@tok@k\endcsname{\let\PY@bf=\textbf\def\PY@tc##1{\textcolor[rgb]{0.00,0.50,0.00}{##1}}}
\expandafter\def\csname PY@tok@kp\endcsname{\def\PY@tc##1{\textcolor[rgb]{0.00,0.50,0.00}{##1}}}
\expandafter\def\csname PY@tok@kt\endcsname{\def\PY@tc##1{\textcolor[rgb]{0.69,0.00,0.25}{##1}}}
\expandafter\def\csname PY@tok@o\endcsname{\def\PY@tc##1{\textcolor[rgb]{0.40,0.40,0.40}{##1}}}
\expandafter\def\csname PY@tok@ow\endcsname{\let\PY@bf=\textbf\def\PY@tc##1{\textcolor[rgb]{0.67,0.13,1.00}{##1}}}
\expandafter\def\csname PY@tok@nb\endcsname{\def\PY@tc##1{\textcolor[rgb]{0.00,0.50,0.00}{##1}}}
\expandafter\def\csname PY@tok@nf\endcsname{\def\PY@tc##1{\textcolor[rgb]{0.00,0.00,1.00}{##1}}}
\expandafter\def\csname PY@tok@nc\endcsname{\let\PY@bf=\textbf\def\PY@tc##1{\textcolor[rgb]{0.00,0.00,1.00}{##1}}}
\expandafter\def\csname PY@tok@nn\endcsname{\let\PY@bf=\textbf\def\PY@tc##1{\textcolor[rgb]{0.00,0.00,1.00}{##1}}}
\expandafter\def\csname PY@tok@ne\endcsname{\let\PY@bf=\textbf\def\PY@tc##1{\textcolor[rgb]{0.82,0.25,0.23}{##1}}}
\expandafter\def\csname PY@tok@nv\endcsname{\def\PY@tc##1{\textcolor[rgb]{0.10,0.09,0.49}{##1}}}
\expandafter\def\csname PY@tok@no\endcsname{\def\PY@tc##1{\textcolor[rgb]{0.53,0.00,0.00}{##1}}}
\expandafter\def\csname PY@tok@nl\endcsname{\def\PY@tc##1{\textcolor[rgb]{0.63,0.63,0.00}{##1}}}
\expandafter\def\csname PY@tok@ni\endcsname{\let\PY@bf=\textbf\def\PY@tc##1{\textcolor[rgb]{0.60,0.60,0.60}{##1}}}
\expandafter\def\csname PY@tok@na\endcsname{\def\PY@tc##1{\textcolor[rgb]{0.49,0.56,0.16}{##1}}}
\expandafter\def\csname PY@tok@nt\endcsname{\let\PY@bf=\textbf\def\PY@tc##1{\textcolor[rgb]{0.00,0.50,0.00}{##1}}}
\expandafter\def\csname PY@tok@nd\endcsname{\def\PY@tc##1{\textcolor[rgb]{0.67,0.13,1.00}{##1}}}
\expandafter\def\csname PY@tok@s\endcsname{\def\PY@tc##1{\textcolor[rgb]{0.73,0.13,0.13}{##1}}}
\expandafter\def\csname PY@tok@sd\endcsname{\let\PY@it=\textit\def\PY@tc##1{\textcolor[rgb]{0.73,0.13,0.13}{##1}}}
\expandafter\def\csname PY@tok@si\endcsname{\let\PY@bf=\textbf\def\PY@tc##1{\textcolor[rgb]{0.73,0.40,0.53}{##1}}}
\expandafter\def\csname PY@tok@se\endcsname{\let\PY@bf=\textbf\def\PY@tc##1{\textcolor[rgb]{0.73,0.40,0.13}{##1}}}
\expandafter\def\csname PY@tok@sr\endcsname{\def\PY@tc##1{\textcolor[rgb]{0.73,0.40,0.53}{##1}}}
\expandafter\def\csname PY@tok@ss\endcsname{\def\PY@tc##1{\textcolor[rgb]{0.10,0.09,0.49}{##1}}}
\expandafter\def\csname PY@tok@sx\endcsname{\def\PY@tc##1{\textcolor[rgb]{0.00,0.50,0.00}{##1}}}
\expandafter\def\csname PY@tok@m\endcsname{\def\PY@tc##1{\textcolor[rgb]{0.40,0.40,0.40}{##1}}}
\expandafter\def\csname PY@tok@gh\endcsname{\let\PY@bf=\textbf\def\PY@tc##1{\textcolor[rgb]{0.00,0.00,0.50}{##1}}}
\expandafter\def\csname PY@tok@gu\endcsname{\let\PY@bf=\textbf\def\PY@tc##1{\textcolor[rgb]{0.50,0.00,0.50}{##1}}}
\expandafter\def\csname PY@tok@gd\endcsname{\def\PY@tc##1{\textcolor[rgb]{0.63,0.00,0.00}{##1}}}
\expandafter\def\csname PY@tok@gi\endcsname{\def\PY@tc##1{\textcolor[rgb]{0.00,0.63,0.00}{##1}}}
\expandafter\def\csname PY@tok@gr\endcsname{\def\PY@tc##1{\textcolor[rgb]{1.00,0.00,0.00}{##1}}}
\expandafter\def\csname PY@tok@ge\endcsname{\let\PY@it=\textit}
\expandafter\def\csname PY@tok@gs\endcsname{\let\PY@bf=\textbf}
\expandafter\def\csname PY@tok@gp\endcsname{\let\PY@bf=\textbf\def\PY@tc##1{\textcolor[rgb]{0.00,0.00,0.50}{##1}}}
\expandafter\def\csname PY@tok@go\endcsname{\def\PY@tc##1{\textcolor[rgb]{0.53,0.53,0.53}{##1}}}
\expandafter\def\csname PY@tok@gt\endcsname{\def\PY@tc##1{\textcolor[rgb]{0.00,0.27,0.87}{##1}}}
\expandafter\def\csname PY@tok@err\endcsname{\def\PY@bc##1{\setlength{\fboxsep}{0pt}\fcolorbox[rgb]{1.00,0.00,0.00}{1,1,1}{\strut ##1}}}
\expandafter\def\csname PY@tok@kc\endcsname{\let\PY@bf=\textbf\def\PY@tc##1{\textcolor[rgb]{0.00,0.50,0.00}{##1}}}
\expandafter\def\csname PY@tok@kd\endcsname{\let\PY@bf=\textbf\def\PY@tc##1{\textcolor[rgb]{0.00,0.50,0.00}{##1}}}
\expandafter\def\csname PY@tok@kn\endcsname{\let\PY@bf=\textbf\def\PY@tc##1{\textcolor[rgb]{0.00,0.50,0.00}{##1}}}
\expandafter\def\csname PY@tok@kr\endcsname{\let\PY@bf=\textbf\def\PY@tc##1{\textcolor[rgb]{0.00,0.50,0.00}{##1}}}
\expandafter\def\csname PY@tok@bp\endcsname{\def\PY@tc##1{\textcolor[rgb]{0.00,0.50,0.00}{##1}}}
\expandafter\def\csname PY@tok@fm\endcsname{\def\PY@tc##1{\textcolor[rgb]{0.00,0.00,1.00}{##1}}}
\expandafter\def\csname PY@tok@vc\endcsname{\def\PY@tc##1{\textcolor[rgb]{0.10,0.09,0.49}{##1}}}
\expandafter\def\csname PY@tok@vg\endcsname{\def\PY@tc##1{\textcolor[rgb]{0.10,0.09,0.49}{##1}}}
\expandafter\def\csname PY@tok@vi\endcsname{\def\PY@tc##1{\textcolor[rgb]{0.10,0.09,0.49}{##1}}}
\expandafter\def\csname PY@tok@vm\endcsname{\def\PY@tc##1{\textcolor[rgb]{0.10,0.09,0.49}{##1}}}
\expandafter\def\csname PY@tok@sa\endcsname{\def\PY@tc##1{\textcolor[rgb]{0.73,0.13,0.13}{##1}}}
\expandafter\def\csname PY@tok@sb\endcsname{\def\PY@tc##1{\textcolor[rgb]{0.73,0.13,0.13}{##1}}}
\expandafter\def\csname PY@tok@sc\endcsname{\def\PY@tc##1{\textcolor[rgb]{0.73,0.13,0.13}{##1}}}
\expandafter\def\csname PY@tok@dl\endcsname{\def\PY@tc##1{\textcolor[rgb]{0.73,0.13,0.13}{##1}}}
\expandafter\def\csname PY@tok@s2\endcsname{\def\PY@tc##1{\textcolor[rgb]{0.73,0.13,0.13}{##1}}}
\expandafter\def\csname PY@tok@sh\endcsname{\def\PY@tc##1{\textcolor[rgb]{0.73,0.13,0.13}{##1}}}
\expandafter\def\csname PY@tok@s1\endcsname{\def\PY@tc##1{\textcolor[rgb]{0.73,0.13,0.13}{##1}}}
\expandafter\def\csname PY@tok@mb\endcsname{\def\PY@tc##1{\textcolor[rgb]{0.40,0.40,0.40}{##1}}}
\expandafter\def\csname PY@tok@mf\endcsname{\def\PY@tc##1{\textcolor[rgb]{0.40,0.40,0.40}{##1}}}
\expandafter\def\csname PY@tok@mh\endcsname{\def\PY@tc##1{\textcolor[rgb]{0.40,0.40,0.40}{##1}}}
\expandafter\def\csname PY@tok@mi\endcsname{\def\PY@tc##1{\textcolor[rgb]{0.40,0.40,0.40}{##1}}}
\expandafter\def\csname PY@tok@il\endcsname{\def\PY@tc##1{\textcolor[rgb]{0.40,0.40,0.40}{##1}}}
\expandafter\def\csname PY@tok@mo\endcsname{\def\PY@tc##1{\textcolor[rgb]{0.40,0.40,0.40}{##1}}}
\expandafter\def\csname PY@tok@ch\endcsname{\let\PY@it=\textit\def\PY@tc##1{\textcolor[rgb]{0.25,0.50,0.50}{##1}}}
\expandafter\def\csname PY@tok@cm\endcsname{\let\PY@it=\textit\def\PY@tc##1{\textcolor[rgb]{0.25,0.50,0.50}{##1}}}
\expandafter\def\csname PY@tok@cpf\endcsname{\let\PY@it=\textit\def\PY@tc##1{\textcolor[rgb]{0.25,0.50,0.50}{##1}}}
\expandafter\def\csname PY@tok@c1\endcsname{\let\PY@it=\textit\def\PY@tc##1{\textcolor[rgb]{0.25,0.50,0.50}{##1}}}
\expandafter\def\csname PY@tok@cs\endcsname{\let\PY@it=\textit\def\PY@tc##1{\textcolor[rgb]{0.25,0.50,0.50}{##1}}}

\def\PYZbs{\char`\\}
\def\PYZus{\char`\_}
\def\PYZob{\char`\{}
\def\PYZcb{\char`\}}
\def\PYZca{\char`\^}
\def\PYZam{\char`\&}
\def\PYZlt{\char`\<}
\def\PYZgt{\char`\>}
\def\PYZsh{\char`\#}
\def\PYZpc{\char`\%}
\def\PYZdl{\char`\$}
\def\PYZhy{\char`\-}
\def\PYZsq{\char`\'}
\def\PYZdq{\char`\"}
\def\PYZti{\char`\~}
% for compatibility with earlier versions
\def\PYZat{@}
\def\PYZlb{[}
\def\PYZrb{]}
\makeatother


    % Exact colors from NB
    \definecolor{incolor}{rgb}{0.0, 0.0, 0.5}
    \definecolor{outcolor}{rgb}{0.545, 0.0, 0.0}



    
    % Prevent overflowing lines due to hard-to-break entities
    \sloppy 
    % Setup hyperref package
    \hypersetup{
      breaklinks=true,  % so long urls are correctly broken across lines
      colorlinks=true,
      urlcolor=urlcolor,
      linkcolor=linkcolor,
      citecolor=citecolor,
      }
    % Slightly bigger margins than the latex defaults
    
    \geometry{verbose,tmargin=1in,bmargin=1in,lmargin=1in,rmargin=1in}
    
    

    \begin{document}
    
    
    \maketitle
    
    

    
    

    \hypertarget{how-i-published-my-own-python-package-and-how-you-can-too}{%
\section{How I published my own Python package and how YOU can
too!}\label{how-i-published-my-own-python-package-and-how-you-can-too}}

\begin{center}\rule{0.5\linewidth}{\linethickness}\end{center}

https://pypi.org/project/bs-ds/ ‡

\texttt{pip\ install\ bs\_ds} \_\_\_

‡ \emph{Note: BroadSteel DataScience is: inspired by, an homage to,
legally distinct from, and in no way representative of} \textbf{Flatiron
School's Online
\href{https://flatironschool.com/career-courses/data-science-bootcamp/online/}{Data
Science bootcamp program.}}

\hypertarget{in-this-post-we-will}{%
\subsection{\texorpdfstring{In this post, we
will:}{In this post, we will: }}\label{in-this-post-we-will}}

\hypertarget{walk-you-through-how-i-made-and-published-my-own-pip-installable-python-package-called-broadsteel_datascience-a.k.a.-bs_ds---using-the-python-package-cookiecutter-to-set-up-the-infrastructure-for-a-professional-quality-pythin-package.}{%
\paragraph{\texorpdfstring{1) \textbf{Walk you through how I made and
published my own pip-installable Python package} called
BroadSteel\_DataScience (A.K.A.
\href{https://pypi.org/project/bs-ds/}{\texttt{bs\_ds}}) - using the
python package \texttt{cookiecutter} to set up the infrastructure for a
professional quality Pythin
package.}{1) Walk you through how I made and published my own pip-installable Python package  called BroadSteel\_DataScience (A.K.A. bs\_ds) - using the python package cookiecutter to set up the infrastructure for a professional quality Pythin package.}}\label{walk-you-through-how-i-made-and-published-my-own-pip-installable-python-package-called-broadsteel_datascience-a.k.a.-bs_ds---using-the-python-package-cookiecutter-to-set-up-the-infrastructure-for-a-professional-quality-pythin-package.}}

\begin{itemize}
\tightlist
\item
  incuding \href{https://bs-ds.readthedocs.io/en/latest/}{auto-generated
  documentation with Sphinx}
\item
  automatic build-testing and deployment with
  \href{https://travis-ci.org/}{travis-ci.org.} 
\end{itemize}

\begin{enumerate}
\def\labelenumi{\arabic{enumi})}
\setcounter{enumi}{1}
\tightlist
\item
  \textbf{Explain how to share access and control of the github/pypi
  package with collaborators}. 
\item
  **Walk through some of my favorite tools in bs\_ds for EDA and
  prettifying DataFrames for projects and presentations.
\end{enumerate}

\begin{center}\rule{0.5\linewidth}{\linethickness}\end{center}

\hypertarget{butwhy-would-i-bother-you-may-be-wondering}{%
\section{\texorpdfstring{``But\ldots{}\emph{why would I bother?}'', you
may be
wondering\ldots{}}{``But\ldots{}why would I bother?'', you may be wondering\ldots{}}}\label{butwhy-would-i-bother-you-may-be-wondering}}

\hypertarget{lots-of-reasons-mostly-convenience}{%
\subsection{\texorpdfstring{\textbf{Lots of reasons!} (\emph{mostly
convenience})}{Lots of reasons! (mostly convenience)}}\label{lots-of-reasons-mostly-convenience}}

\hypertarget{but-also}{%
\subsubsection{But also\ldots{}}\label{but-also}}

\begin{enumerate}
\def\labelenumi{\arabic{enumi}.}
\tightlist
\item
  \textbf{Becuase you're \emph{sick} of having to copy and paste your
  favorite functions into every new notebook.}

  \begin{itemize}
  \tightlist
  \item
    I work on several computers and also in the cloud. It was always a
    pain to have to log into Dropbox on any new computer and remmeber
    where I saved that notebook that had all of those cool little
    functions I wrote\ldots{}I
  \end{itemize}
\item
  \textbf{Because you collaborate with others and want to ensure you all
  have the same toolkit at your disposal.}

  \begin{itemize}
  \tightlist
  \item
    My collaborator and fellow student,
    \href{https://github.com/MichaelMoravetz}{Michael Moravetz}, and I
    have been working closely together and we wanted an easy way to
    share the cool/helpful tools each of us that we either wrote
    ourselves or were given in our bootcamp lessons.
  \end{itemize}
\item
  \textbf{Because you're instructor tells you that your notebook has WAY
  too many functions up front, and its distracting\ldots{}}

  \begin{itemize}
  \tightlist
  \item
    \textbf{\emph{You:}} ``But aren't I \emph{supposed} to write a lot
    of functions to become a better programmer?''"
  \item
    \textbf{\emph{Me:}} ``Yes\ldots{}you're right\ldots{} BUT when
    someone has to scroll through a lot of functions-as-appetizers
    before geting to your main-course-notebook, they may not have much
    attentional-appetite left.''
  \end{itemize}
\item
  \textbf{Because you're a little OCD and like having ALL of your
  notebook's settings JUST THE RIGHT WAY.}

  \begin{itemize}
  \tightlist
  \item
    pandas.set\_options()
  \item
    HTML/CSS styling, etc.)
  \item
    Matplotlib Params
  \end{itemize}
\item
  \textbf{Finally, becuase you're lazy coder} and just want all of your
  tools imported for you and ready to use whereever you go with as
  little effort as possible.

  \begin{itemize}
  \tightlist
  \item
    \textbf{I'm a big believer in exerting extra up-front-effort in the
    name of future-laziness-convience.}
  \end{itemize}
\end{enumerate}

\begin{center}\rule{0.5\linewidth}{\linethickness}\end{center}

\hypertarget{ok-i-get-it-youre-hopefully-thinking-so-how}{%
\section{\texorpdfstring{``OK, I get it,'' you're (hopefully)
thinking\ldots{} ``So,
\emph{how?}''}{``OK, I get it,'' you're (hopefully) thinking\ldots{} ``So, how?''}}\label{ok-i-get-it-youre-hopefully-thinking-so-how}}

There are simpler ways to go about this, but I am going to show you the
exact tools I use that allow me to have an
automatically-updated-and-deployed package with very little effort (once
its set up). \#\# THE TOOLS I USE: -
\href{https://www.anaconda.com/}{Anaconda-installed Python} -
\href{https://code.visualstudio.com/docs/introvideos/basics}{Microsoft's
Visual Studio Code} - Its optionally installed with Anaconda and has
quickly become my favorite editor. \emph{(Sorry, SublimeText3\ldots{}
It's not you! Its me. I still love you..I'm just not as }\textbf{in
love}* with you as I used to be. But we can still be friends, right?!)*
- \href{https://desktop.github.com/}{GitHub Desktop} - Its SO convenient
and simplifies the workflow and collaboration immensely. -
\href{https://cookiecutter-pypackage.readthedocs.io/en/latest/}{Cookiecutter}
python package - For the total package infrastructure, with the opton of
setting up automation and documentation generation. - \href{}{Google
Colab}, \href{}{Microsoft Azure Notebooks} - For testing the package in
cloud Jupyter notebooks. Just add an `!' to the pip install command in
any cloud-notebook \texttt{!pip\ install\ bs\_ds}

    \hypertarget{how-to-create-your-package-infrastructure-using-cookiecutter}{%
\section{HOW TO CREATE YOUR PACKAGE INFRASTRUCTURE USING
COOKIECUTTER}\label{how-to-create-your-package-infrastructure-using-cookiecutter}}

\hypertarget{please-open-the-official-cookiecutter-tutorial}{%
\subsection{PLEASE OPEN THE OFFICIAL COOKIECUTTER
TUTORIAL:}\label{please-open-the-official-cookiecutter-tutorial}}

\begin{itemize}
\tightlist
\item
  {[} {]} \textbf{Open Cookiecutter's
  \href{https://cookiecutter-pypackage.readthedocs.io/en/latest/tutorial.html}{official
  tutorial from their documentation.}}

  \begin{itemize}
  \tightlist
  \item
    Its very good overall but I suggest doing a few steps a little
    differently. My suggested steps will start with ``REC'D'' to
    indicate deviations from the official steps.
  \end{itemize}
\item
  {[} {]} \textbf{Make sure to reference BOTH my instructions beklow as
  well as the official tutorial.}

  \begin{itemize}
  \tightlist
  \item
    Make \textbf{read my recommended steps first,} as I suggest doing
    some steps earlier than the tutorial does.
  \end{itemize}
\end{itemize}

\hypertarget{my-recommended-steps-for-cutting-your-first-cookies}{%
\subsection{MY RECOMMENDED STEPS FOR CUTTING YOUR FIRST
COOKIES:}\label{my-recommended-steps-for-cutting-your-first-cookies}}

\hypertarget{before-getting-started-create-accounts-for-requireddesired-services}{%
\subsubsection{\texorpdfstring{BEFORE GETTING STARTED, CREATE ACCOUNTS
FOR REQUIRED/DESIRED
SERVICES:}{BEFORE GETTING STARTED, CREATE ACCOUNTS FOR REQUIRED/DESIRED SERVICES: }}\label{before-getting-started-create-accounts-for-requireddesired-services}}

\begin{enumerate}
\def\labelenumi{\arabic{enumi}.}
\tightlist
\item
  {[} {]} \textbf{Go to PyPi.org and ``Register'' a new account.
  {[}REQUIRED{]}} PyPi.org is the offical service that hosts
  pip-installable packages.

  \begin{itemize}
  \tightlist
  \item
    You will need your account name very soon, during the initial
    cookie-cutting step below.
  \item
    You will need your password later, when you are ready to upload your
    package to PyPi. 
  \end{itemize}
\item
  {[} {]} \textbf{Register your github account with www.travis-ci.org.
  {[}OPTIONAL, BUT HIGHLY RECOMMENDED{]}} for automated build testing
  and auto-deployment to PyPi.

  \begin{itemize}
  \tightlist
  \item
    NOTE: Make sure to go to \textbf{www.travis-ci.ORG} (\textbf{NOT}
    travis-ci.\textbf{COM}, which is for commerical, non-open source
    pcakges) 
  \item
    \textbf{Suggestion: This is totally worth it.} It adds a little
    complexity to the cookie-cutting set up process, but it:

    \begin{itemize}
    \tightlist
    \item
      makes updating your package a breeze.
    \item
      makes it easier for others to contribute to your package
    \item
      it will pre-test any Pull Requests so you will already know if the
      code is functional before you merge their code with yours.
    \end{itemize}
  \end{itemize}
\item
  {[} {]} \textbf{Register an account on
  \href{https://readthedocs.org/}{readthedocs.org} {[}OPTIONAL, BUT
  REC'D IF SHARING YOUR WORK{]}}

  \begin{itemize}
  \tightlist
  \item
    Readthedocs will host your generated user documentation for your
    package.
  \item
    Note: Cookiecutter will fill in a lot of the documentation basics
    for you.
  \item
    Note: There is an additional advanced method to auto-generate all
    documentation from docstrings, which I will mention in the tutorial
    below.
  \end{itemize}
\end{enumerate}

\begin{center}\rule{0.5\linewidth}{\linethickness}\end{center}

\hypertarget{recd-step-0.-before-instaling-anything-you-should}{%
\subsubsection{\texorpdfstring{REC'D STEP 0. \textbf{BEFORE INSTALING
ANYTHING}, you
should:}{REC'D STEP 0. BEFORE INSTALING ANYTHING, you should: }}\label{recd-step-0.-before-instaling-anything-you-should}}

\begin{itemize}
\tightlist
\item
  {[} {]} \textbf{Create a new virtual environment}, preferably by
  cloning your current one.

  \begin{itemize}
  \tightlist
  \item
    Anaconda Navigator makes the cloning process easy.

    \begin{itemize}
    \tightlist
    \item
      In Navigator, click on the \texttt{Environments} tab in the
      sidebar.
    \item
      Click on your current enivornment of choice, then click the Clone
      button, and give it a new name.
    \end{itemize}
  \item
    {[} {]} Add this to your Jupyter notebook kernels using
    \texttt{python\ -m\ ipykernel\ install\ -\/-user\ -\/-name\ myenv}
  \end{itemize}
\item
  {[} {]} \textbf{Backup and export your current enviornment} to a .yml
  file, which you can use to re-install your env, if need be.

  \begin{itemize}
  \tightlist
  \item
    For Anaconda environments, open your terminal and activate your
    environment before exporting: \texttt{source\ activate\ env-name}
    \texttt{conda\ env\ export\ \textgreater{}\ my\_environment.yml}

    \begin{itemize}
    \tightlist
    \item
      \emph{where ``env-name'' is the name of the environment you'd like
      to clone and ``my\_environment''.yml is any-name-you'd-like.yml}
    \end{itemize}
  \item
    This will save the .yml into your current directory that can be used
    to install your environment in the future
    using:\texttt{conda\ env\ create\ -f\ my\_environment.yml}
  \end{itemize}
\item
  \textbf{DO NOT SKIP THIS STEP.} I have warned you and I am not
  responsible for any broken environments.While nothing \emph{should}
  break, it's always a GOOD idea to create a new environment for
  creating and installing test packages. Really, I should say its a DUMB
  idea not to. 
\end{itemize}

\hypertarget{recd-step-1-install-cookiecutter-into-your-new-environment.}{%
\subsubsection{REC'D STEP 1: Install cookiecutter into your new
environment.}\label{recd-step-1-install-cookiecutter-into-your-new-environment.}}

\begin{itemize}
\tightlist
\item
  Tutorial ``Step 1: Install Cookiecutter'': Install Cookie Cutter and
  cookie-cut the default template cookiecuter repo.

  \begin{itemize}
  \tightlist
  \item
    {[}x{]} You may ignore the first part of Step 1 (using virtualenv to
    create an env). 
  \item
    {[} {]} Install cookiecutter via pip:
    \texttt{pip\ install\ cookiecutter}
  \end{itemize}
\end{itemize}

\hypertarget{recd-step-2-create-a-new-github-repo-for-your-package-and-clone-it-locally}{%
\subsubsection{REC'D STEP 2: Create a new GitHub repo for your package
and clone it
locally}\label{recd-step-2-create-a-new-github-repo-for-your-package-and-clone-it-locally}}

\textbf{\emph{NOTE: My recommendation deviates from the tutorial. This
will replace ``Step 3: Create a GitHub Repo''.}} - {[} {]} Log into your
your GitHub profile on github.com and Create a New Repository by
clicking the + sign next to your account picture on the top-right of the
page. - {[} {]} Create a New Repository, using the desired name for your
published package for the repo name. - {[} {]} Check
``\texttt{Initialize\ this\ repo\ with\ a\ README}'' (you can't clone an
empty repo). - Leave the rest of the options blank/none.
\texttt{Initialzie\ with\ Add\ a\ .git\ ignore},
\texttt{Add\ a\ license} - Cookiecutter will ask you to choose a license
later in the process. - {[} {]} Clone the new repo to your computer.
(This is the perfect chance to try using
\href{https://desktop.github.com/}{GitHub Desktop}, if you haven't
before. ) - Click Clone or Download and: - Copy the url if you plan on
using your terminal to clone. - OR ``Open in Desktop'' if you've
installed and logged in to the GitHub Desktop App. 

    \hypertarget{official-step-3.-use-the-cookiecutter-command-to-cut-your-first-cookie-template.}{%
\subsubsection{OFFICIAL STEP \#3. Use the cookiecutter command to
cut-your-first-cookie-template.}\label{official-step-3.-use-the-cookiecutter-command-to-cut-your-first-cookie-template.}}

\begin{itemize}
\tightlist
\item
  {[} {]} Activate cloned environment from step \#1, \texttt{cd}into
  your repo's folder.
\item
  {[} {]} Enter the following command to create the template
  infrastructure:
\end{itemize}

\begin{quote}
\texttt{cookiecutter\ https://github.com/audreyr/cookiecutter-pypackage.git}
\end{quote}

\hypertarget{cookiecutter-prompts-and-recommended-answers}{%
\subsubsection{Cookiecutter Prompts and recommended
answers:}\label{cookiecutter-prompts-and-recommended-answers}}

\begin{itemize}
\tightlist
\item
  \textbf{Cookiecutter will ask you several questions during the
  cookie-cutting process,
  \href{https://cookiecutter-pypackage.readthedocs.io/en/latest/prompts.html}{check
  this resouce to see the descriptions for each prompt.}}
\end{itemize}

\hypertarget{the-cookiecutter-options-i-selected-for-bs_ds-were-as-follows}{%
\paragraph{The cookiecutter options I selected for bs\_ds were as
follows:}\label{the-cookiecutter-options-i-selected-for-bs_ds-were-as-follows}}

\begin{itemize}
\tightlist
\item
  \textbf{``project\_slug''}

  \begin{itemize}
  \tightlist
  \item
    should match the name of your new repo from step \#2.
  \item
    It should be something terminal-syntax (no -'s or spaces, etc.)
  \end{itemize}
\item
  \textbf{``project\_name''}

  \begin{itemize}
  \tightlist
  \item
    will be what appears in all of the generated documentation. It can
    have spaces and any characters that you wish.
  \end{itemize}
\item
  \textbf{``use\_pytest''}:

  \begin{itemize}
  \tightlist
  \item
    use default `n'
  \end{itemize}
\item
  \textbf{``use\_pypi\_deployment\_with\_travis''}:

  \begin{itemize}
  \tightlist
  \item
    use `y' for auto-deployment with travis-ci.org (will need an
    account, as described above)
  \end{itemize}
\item
  \textbf{``add\_pyup\_badge''}:

  \begin{itemize}
  \tightlist
  \item
    use default `n'
  \end{itemize}
\item
  \textbf{``Select command\_line\_interface:''}

  \begin{itemize}
  \tightlist
  \item
    I suggest option 2 for No command-line interface.
  \end{itemize}
\item
  \textbf{``Select open source license''}

  \begin{itemize}
  \tightlist
  \item
    This is an important choice that determines what people are allowed
    to do with your code with or without your permission.

    \begin{itemize}
    \tightlist
    \item
      Consult https://choosealicense.com/ (github website explaining
      licenses) for information.
    \end{itemize}
  \item
    Note: bs\_ds is published using option 5 - GNU General Public
    License v3, which choosealicense.com defines as: \textgreater{}``The
    GNU GPLv3 also lets people do almost anything they want with your
    project, except to distribute closed source versions.''
  \end{itemize}
\item
  \textbf{Cookiecutter will then create a new folder inside of your main
  repo folder, whose name is determined by the ``project\_slug'' entered
  above.}
\end{itemize}

\hypertarget{step-3b-required-if-you-followed-recd-step-2-and-created-the-repo-first}{%
\paragraph{STEP \#3B {[}REQUIRED if you followed REC'D STEP \#2 and
created the repo
first{]}:}\label{step-3b-required-if-you-followed-recd-step-2-and-created-the-repo-first}}

\begin{itemize}
\tightlist
\item
  If you followed my REC'D STEP \#2, you main repo folder should now
  contain:

  \begin{itemize}
  \item
    a README.md file
  \item
    a ``.git'' folder
  \item
    a new subfolder whose name == project\_slug entered above (I will
    refer to as ``slug folder \#1'')

    \begin{itemize}
    \tightlist
    \item
      Inside of the project\_slug folder, you should find:

      \begin{itemize}
      \tightlist
      \item
        a " .github" folder
      \item
        a ``docs'' folder
      \item
        a ``tests'' folder
      \item
        ANOTHER folder whose name == project\_slug (I will refer to as
        ``slug folder \#2'')
      \item
        and a text file called ``requirements\_dev.txt'', several .rst
        files, setup.py, setup.cfg, and several other files.
      \end{itemize}
    \end{itemize}
  \item
    {[} {]} \textbf{Move(or cut) all of the contents from inside slug
    folder \#1 and move/paste them into the main repo folder.}
  \item
    After moving the contents to the main repo folder, there should be :

    \begin{itemize}
    \item
      A project\_slug folder (which is actually slug\_folder\#2 now),
    \item
      requirements-dev.txt
    \item
      and the .rst and setup files originally from slug folder\#2.
    \item
      Inisde of the project\_slug folder, there should only be 2 files
      and 0 folders:

      \begin{itemize}
      \tightlist
      \item
        \textbf{init}.py
      \item
        project\_slug.py
      \end{itemize}
    \end{itemize}
  \end{itemize}
\item
  \textbf{If so congratulations! You have the infrastructure properly
  installed!}
\end{itemize}

    \hypertarget{official-step-4-install-dev-requirements}{%
\subsubsection{Official Step 4: Install Dev
Requirements}\label{official-step-4-install-dev-requirements}}

\begin{itemize}
\item
  In your terminal, make sure you are still located in the main repo
  folder, which contains \textbf{requirements-dev.txt}
\item
  Make sure you are still using your newly cloned environment, then
  enter: \texttt{pip\ install\ -r\ requirements\_dev.txt} \_\_\_
\item
  {[} {]} This is a decent place to take a moment to commit your changes
  and push to your github repo. \_\_\_
\end{itemize}

\hypertarget{official-step-5-set-up-travis-ci}{%
\subsubsection{Official Step 5: Set Up Travis
CI}\label{official-step-5-set-up-travis-ci}}

\begin{itemize}
\tightlist
\item
  {[} {]} \textbf{In order to follow the offical step 5, you will need
  to install Travis CLI tool, which requires Ruby.}
  \href{https://cookiecutter-pypackage.readthedocs.io/en/latest/travis_pypi_setup.html\#travis-pypi-setup}{Instructions
  are located here and are OS-specific},

  \begin{itemize}
  \tightlist
  \item
    For MacOS, they recommend using the Homebrew travis package:

    \begin{itemize}
    \tightlist
    \item
      \texttt{brew\ install\ travis}
    \end{itemize}
  \item
    For windows, you will need to install ruby and then use
    \texttt{gem\ install} to install travis.

    \begin{itemize}
    \tightlist
    \item
      {[} {]} \href{http://www.ruby-lang.org/en/downloads/}{Install
      Ruby} (if not already installed on your system)
    \item
      {[} {]} Install Travis CLI tool: (See the OS-specifc instructions
      directly above)

      \begin{itemize}
      \tightlist
      \item
        After Ruby is installed, enter the following command to install
        Travis CLI tool.
        \texttt{gem\ install\ travis\ -v\ 1.8.10\ -\/-no-rdoc\ -\/-no-ri}
      \end{itemize}
    \end{itemize}
  \end{itemize}
\item
  {[} {]} \textbf{Once Travis CLI is installed, you may continue to
  follow the
  \href{https://cookiecutter-pypackage.readthedocs.io/en/latest/tutorial.html\#step-5-set-up-travis-ci}{official
  tutorial instructions for step \#5}}

  \begin{itemize}
  \tightlist
  \item
    NOTE: Here is where you will need to have your password for PyPi
    available.

    \begin{itemize}
    \tightlist
    \item
      \textbf{CAUTION:} When entering your PyPi username and password in
      the terminal, there will be \textbf{NO VISUAL INDICATOR that you
      have typed your password.} 
    \item
      There are no characters displayed and no dots or placeholders to
      indicated the \# of characters entered, so \textbf{carefully enter
      your password when prompted and press enter.}
    \end{itemize}
  \item
    \textbf{TROUBLESHOOTING NOTE: If Travis doesn't does not ask for
    your password after entering username:}

    \begin{itemize}
    \tightlist
    \item
      I experienced an issue when attempting to follow step \#5, after
      entering the \texttt{travis\ encrypt\ -\/-add\ deploy.password}
      command, you should first be prompted for your username, then your
      password.
    \item
      I use Git Bash for my main terminal on Windows and for some reason
      \textbf{Travis would hang after I entered my username and would
      never ask me for password.}

      \begin{itemize}
      \tightlist
      \item
        I got around the issue by \textbf{using the normal windows cmd
        prompt for this step instead of using GitBash.} (This is a
        one-time step that will encrypt your password and store it in a
        config file so you never have to enter it again.)
      \end{itemize}
    \end{itemize}
  \end{itemize}
\end{itemize}

\hypertarget{official-step-6-set-up-readthedocs}{%
\subsubsection{Official Step 6: Set Up
ReadTheDocs}\label{official-step-6-set-up-readthedocs}}

\begin{itemize}
\tightlist
\item
  {[} {]} Follow the
  \href{https://cookiecutter-pypackage.readthedocs.io/en/latest/tutorial.html\#step-6-set-up-readthedocs}{official
  tutorial step 6} for setting up documentation on readthedocs.org.
\end{itemize}

\hypertarget{official-step-7set-up-pyup.io}{%
\subsubsection{\texorpdfstring{\sout{Official Step 7:Set Up
pyup.io}}{Official Step 7:Set Up pyup.io}}\label{official-step-7set-up-pyup.io}}

\begin{itemize}
\tightlist
\item
  \textbf{Short Version: This is an added level of complexity that I
  chose to skip for myself and recommend you do the same for now.}
\item
  I recommended skipping setting up pyup.io during the cookiecutter
  prompt responses above.

  \begin{itemize}
  \tightlist
  \item
    This service would alert you when any of the required python
    packages that are your package needs to run have been updated, so
    that you can update the versions in your installation requirements\\
    \textbf{\emph{ }}
  \end{itemize}
\end{itemize}

\textbf{SIDEBAR: As of now, you may realize that you have not actually
added any code to your python package, and yet the next official step is
to release on PyPi.}

\begin{itemize}
\tightlist
\item
  If you'd like to add some of your code before submitting your package
  to PyPi, jump down to the ``Adding Your Code / Editing your package''
  section ( after the official instructions).
\end{itemize}

\begin{center}\rule{0.5\linewidth}{\linethickness}\end{center}

\begin{center}\rule{0.5\linewidth}{\linethickness}\end{center}

\hypertarget{official-step-8-release-on-pypi}{%
\subsubsection{Official Step 8: Release on
PyPi}\label{official-step-8-release-on-pypi}}

\hypertarget{one-last-annoying-first-time-only-hurdle-and-then-youre-on-your-way-to-automated-deployment-for-the-future}{%
\paragraph{One last annoying, first-time-only hurdle and then you're on
your way to automated deployment for the
future!}\label{one-last-annoying-first-time-only-hurdle-and-then-youre-on-your-way-to-automated-deployment-for-the-future}}

\begin{itemize}
\tightlist
\item
  Travis-CI will automate the process for generating the distribution
  files for your package and uploading them to PyPi, BUT it cannot
  CREATE a NEW package that doesn't already exist on PyPi's servers.
\item
  {[} {]} To \emph{register} your new package with PyPi for the very
  first version, you must \textbf{manually create and upload the very
  first version} of your package.
  \href{https://packaging.python.org/tutorials/packaging-projects/\#generating-distribution-archives}{Official
  Python instructions for ``generating distribution archives'',
  summarized below}

  \begin{itemize}
  \tightlist
  \item
    Briefly, from inside the main folder of your repo (that contains the
    setup.py file):

    \begin{enumerate}
    \def\labelenumi{\arabic{enumi}.}
    \tightlist
    \item
      {[} {]} In your terminal (in your cloned environment), make sure
      you have the current setuptools installed:
      \texttt{python3\ -m\ pip\ install\ -\/-user\ -\/-upgrade\ setuptools\ wheel}
    \item
      {[} {]} Build the current version of your package
      \texttt{python3\ setup.py\ sdist\ bdist\_wheel}
    \item
      {[} {]} Install the tool for uploading to pypi, twine:
      \texttt{python3\ -m\ pip\ install\ -\/-user\ -\/-upgrade\ twine}
    \item
      {[} {]} Upload the distribution files created above (inside a new
      folder called dist/) \texttt{twine\ upload\ dist/*}

      \begin{itemize}
      \tightlist
      \item
        {[} {]} When prompted, enter your PyPi.org username and
        password.
      \end{itemize}
    \item
      Thats it! You can go to PyPi.org, log into your account and you
      should see your package appear under ``Your Projects''\\
    \end{enumerate}

    \begin{itemize}
    \item
      **After a couple moments, your package should be available on pip.
      \texttt{pip\ install\ my\_package\_name} to install locally or
      \texttt{!pip\ install\ my\_package\_name} to install in a cloud
      notebook.
    \item
      TROUBLESHOOTING NOTE:

      \begin{itemize}
      \tightlist
      \item
        For me, using \texttt{python3} for the above commnads did not
        work. I simply had to change \texttt{python3} to just
        \texttt{python}

        \begin{itemize}
        \tightlist
        \item
          Example:
          \texttt{python\ -m\ pip\ install\ -\/-user\ -\/-upgrade\ setuptools\ wheel}
        \end{itemize}
      \item
        If this doesn't fix it for you, you may need to update your
        systems Path variable (basically a list that tells your computer
        all of the locations on your PC where you may have
        scripts/functions saved to run from your terminal).

        \begin{itemize}
        \tightlist
        \item
          For Windows,
          \href{https://geek-university.com/python/add-python-to-the-windows-path/}{check
          this article} for instructions on how to add python to your
          system path.
        \item
          For Mac,
          \href{https://programwithus.com/learn-to-code/install-python3-mac/}{try
          this article's suggestions}
        \end{itemize}
      \end{itemize}
    \end{itemize}
  \end{itemize}
\end{itemize}

    \begin{center}\rule{0.5\linewidth}{\linethickness}\end{center}

\begin{center}\rule{0.5\linewidth}{\linethickness}\end{center}

\hypertarget{adding-your-code-editing-your-packagemodules}{%
\section{Adding Your Code / Editing your
package/modules}\label{adding-your-code-editing-your-packagemodules}}

\begin{itemize}
\tightlist
\item
  When working on your package/modules, I highly recommend using
  \textbf{Microsoft Visual Studio Code.}

  \begin{itemize}
  \tightlist
  \item
    Visual Studio was likely installed with Anaconda, but if it wasn't.
    Open Anaconda Navigator, and look for Visual Studio code on the Home
    tab, in the same section as Jupyter Lab and Jupyter Notebooks.
  \end{itemize}
\item
  The easiest way to manage all of your package's setup files and
  modules is to the the File \textgreater{} Open Folder option and
  select your repo's main folder.
\end{itemize}

\hypertarget{important-files-you-will-want-to-edit}{%
\subsection{Important Files You Will Want to
Edit:}\label{important-files-you-will-want-to-edit}}

\hypertarget{your-package-init-and-module-.py-files}{%
\subsubsection{\texorpdfstring{Your package \textbf{init} and module .py
files:}{Your package init and module .py files:}}\label{your-package-init-and-module-.py-files}}

\begin{itemize}
\tightlist
\item
  Inside of your main repo folder, you should have your project\_slug
  folder (where project\_slug = your package's name)

  \begin{itemize}
  \tightlist
  \item
    There should be 2 files inside that folder: \textbf{init}.py, and
    project\_slug.py .

    \begin{itemize}
    \item
      \textbf{\textbf{init}.py is the most critical file of your
      package.} When you import your package, you are actually running
      the \textbf{init}.py file and importing the functions inside it.
    \item
      \textbf{The simplest way to add your own functions is to add them
      to the \textbf{init}.py file.}

      \begin{itemize}
      \tightlist
      \item
        When you use \texttt{import\ package\_name}:

        \begin{itemize}
        \tightlist
        \item
          The functions and commands contained in your \textbf{init}.py
          file will be imported under your package's name.
        \item
          Example:\texttt{package\_name.some\_function()}
        \end{itemize}
      \item
        As with all python packages, you can assign it a short handle to
        make accessing your functions less tedious:

        \begin{itemize}
        \tightlist
        \item
          Example
          \texttt{import\ package\_name\ as\ pn\ \ pn.some\_function()}
        \end{itemize}
      \item
        If you use \texttt{from\ package\_name\ import\ *}:

        \begin{itemize}
        \item
          All of the functions inside of the init file will be available
          without needing to specify the package.
        \item
          Example:
          \texttt{from\ package\_name\ import\ *}\texttt{some\_function()}
        \end{itemize}
      \end{itemize}
    \item
      **The more advanced way to add your own functions is to add them
      as a sub-module.

      \begin{itemize}
      \tightlist
      \item
        The project\_slug.py file is actually a submodule of your
        package, but shares the same name.

        \begin{itemize}
        \tightlist
        \item
          For bs\_ds, we have many functions stored inside of the
          package submodule:

          \begin{itemize}
          \tightlist
          \item
            Which is accessed by bs\_ds.bs\_ds which is the
            (package\_name).(submodule\_name)
          \item
            The package name is essentially the project\_slug folder and
            then the submodule name is specifying which .py file (INSIDE
            of that folder) should be imported.
          \end{itemize}
        \end{itemize}
      \end{itemize}
    \end{itemize}
  \end{itemize}
\end{itemize}

\hypertarget{setup.py}{%
\subsubsection{Setup.py}\label{setup.py}}

\begin{itemize}
\tightlist
\item
  {[} {]} Adding dependencies to be installed with your package:

  \begin{itemize}
  \tightlist
  \item
    At the top of the file, you will see an empty list called
    requirements \texttt{requirements\ =\ {[}\ {]}}
  \item
    Add any packages that you would like to be installed with your
    package.

    \begin{itemize}
    \tightlist
    \item
      If the user is missing any of these pip will install them as well.
      \texttt{requirements\ =\ {[}\textquotesingle{}numpy\textquotesingle{},\textquotesingle{}pandas\textquotesingle{},\textquotesingle{}scikit-learn\textquotesingle{},\textquotesingle{}matplotlib\textquotesingle{},\textquotesingle{}scipy\textquotesingle{},\textquotesingle{}pprint\textquotesingle{}{]}}
    \end{itemize}
  \end{itemize}
\end{itemize}

\hypertarget{travis.yml}{%
\subsubsection{travis.yml}\label{travis.yml}}

\textbf{travis.yml controls the build testing and deployment process.} -
At the top of the file, there is a list of python versions (3.6, 3.5,
etc.) - {[} {]} You may want to remove versions of python that your
package cannot support. - For example, f-string formatting wasn't added
until Python 3.6
\texttt{print(f"Print\ the\ \{variable\_contents\}\textquotesingle{})} -
Otherwise, your build will fail when travis tests the older version of
python, since you used functions that were not compatible with old
versions. - bs\_ds only supports 3.6 at the moment. - At the bottom of
the file, there is a \texttt{deploy:} section. - I personally had
difficult using \texttt{-\/-tags} in order to trigger the deployment of
bs\_ds. - I removed the \texttt{tags:true} line under \texttt{on:},
which is at the bottom of the \texttt{deploy:} section.

\hypertarget{setup.cfg}{%
\subsubsection{setup.cfg}\label{setup.cfg}}

\begin{itemize}
\tightlist
\item
  {[} {]} If you removed the
  \texttt{tags:true\textasciigrave{}\textasciigrave{}\ line\ from\ travis.yml,\ you\ should\ also\ remove:}tag
  = True``under {[}bumpversion{]}
\end{itemize}

\begin{Shaded}
\begin{Highlighting}[]
\NormalTok{[bumpversion]}
\NormalTok{current_version }\OperatorTok{=} \DecValTok{0}\NormalTok{.}\FloatTok{1.0}
\NormalTok{commit }\OperatorTok{=} \VariableTok{True}
\NormalTok{tag }\OperatorTok{=} \VariableTok{True}
\end{Highlighting}
\end{Shaded}

\begin{verbatim}
- This means that instead of waiting for a special --tagged commit ...
\end{verbatim}

{[}!{]} \_\_\_ \# Updating and Deploying Your Package - For a checklist
of steps to deploy an updated version of your package,
\href{https://gist.github.com/audreyr/5990987}{see this file.}

\hypertarget{to-deploy-an-updated-version-of-your-package}{%
\subsection{To deploy an updated version of your
package:}\label{to-deploy-an-updated-version-of-your-package}}

\begin{enumerate}
\def\labelenumi{\arabic{enumi}.}
\tightlist
\item
  {[} {]} Debug your modules.
\item
  {[} {]} Save all updated files and commit them to your repo.
\item
  {[} {]} Bump the version number and commit again.
\item
  {[} {]} Push the repo back to git.
\end{enumerate}

\hypertarget{to-check-on-the-status-of-your-package}{%
\subsection{To check on the status of your
package:}\label{to-check-on-the-status-of-your-package}}

\hypertarget{checking-build-tests-on-travis-ci.org}{%
\subsubsection{Checking build tests on
travis-CI.org}\label{checking-build-tests-on-travis-ci.org}}

\hypertarget{checking-documentation-creation-on-readthedocs.org}{%
\subsubsection{Checking documentation creation on
readthedocs.org}\label{checking-documentation-creation-on-readthedocs.org}}

\hypertarget{debug-your-modules-before-committing-to-save-time-instead-of-waiting-for-travis-ci}{%
\subsubsection{Debug your modules before committing (to save time
instead of waiting for
Travis-CI)}\label{debug-your-modules-before-committing-to-save-time-instead-of-waiting-for-travis-ci}}

\begin{itemize}
\tightlist
\item
  Visual Studio Code has a very handy Debug feature, which you can
  access from the sidebar (its the symbol with the bug on it).
\item
  On the top of the sidebar that appears, there is a dropdown menu with
  a green play button.

  \begin{itemize}
  \tightlist
  \item
    Open the file you want to test (testing \textbf{init}.py is always
    recommended, but you should test any modules that have been updated.
  \item
    From this menu, select Python Module. \#\#\# UPDATE THE VERSION \#:
  \item
    PyPi.org will only accept new versions of your package if it has a
    unique version number.

    \begin{itemize}
    \tightlist
    \item
      It does not matter if your code has changed, PyPi will not publish
      it if the version number already exists.
    \item
      To change the version \#s for your package, we use ``bumpversion''
    \end{itemize}
  \end{itemize}
\end{itemize}

\hypertarget{bumpversion}{%
\paragraph{bumpversion:}\label{bumpversion}}

\begin{itemize}
\tightlist
\item
  The version number for your package is located in 3 file locations:

  \begin{itemize}
  \tightlist
  \item
    setup.cfg
  \item
    setup.py
  \item
    \textbf{init}.py
  \end{itemize}
\item
  \textbf{bumpversion} will increment all 3 locations when you enter a
  bumpversion command in your terminal.

  \begin{itemize}
  \tightlist
  \item
    bumpversion has understands 3 types of updates: major, minor, and
    patch.

    \begin{itemize}
    \tightlist
    \item
      For example, let's say your package is currently v 0.1.0

      \begin{itemize}
      \tightlist
      \item
        \texttt{bumpversion\ major}

        \begin{itemize}
        \tightlist
        \item
          Increment version \#'s by 1's
        \item
          v 0.1.0 is bumped to v 1.0.0
        \end{itemize}
      \item
        \texttt{bumpversion\ minor}

        \begin{itemize}
        \tightlist
        \item
          increments version by 0.1
        \item
          v 0.1.0 is bumped to v 0.2.0
        \end{itemize}
      \item
        \texttt{bumpversion\ patch}

        \begin{itemize}
        \tightlist
        \item
          increments version by 0.0.1
        \item
          v 0.1.0 is bumped to v 0.1.1
        \end{itemize}
      \end{itemize}
    \end{itemize}
  \item
    Before entering the bumpversion command, you must commit any changes
    you've made to your repo.

    \begin{itemize}
    \tightlist
    \item
      bumpversion will return an error if you try to bump without
      committing first.
    \end{itemize}
  \item
    {[} {]} To increment your package's version \#:

    \begin{itemize}
    \tightlist
    \item
      Commit any changes you've made for your new version. (note: you do
      not need to \texttt{git\ push} yet. Committing the changes is
      sufficient to appease bumpversion)
    \end{itemize}
  \item
    {[} {]} Enter the appropriate bumpversion command depending on how
    much you'd like to increase the version \#.
  \item
    {[} {]} Push the updated repo. If you removed the tags:true entries
    as suggested above, Travis-CI will automatically build test and
    attempt to deploy any commits to your package.
  \end{itemize}
\item
  NOTE: While this may sound risky, its actually not, since PyPi will
  not deploy any packages with the same version \#.

  \begin{itemize}
  \tightlist
  \item
    As long as you do not bumpversion, Travis will \emph{test} your
    updated package, but it will \emph{fail to deploy it}. since PyPi
    already has a pre-existing distribution for that version \#.
  \item
    Documentation Note:

    \begin{itemize}
    \tightlist
    \item
      Readthedocs.org will test and update your documentation for ANY
      commit. So if you only need to update an aspect of the doc's, you
      can simply change the settings and push your repo \emph{without}
      having to bumpversion.
    \end{itemize}
  \end{itemize}
\end{itemize}


    % Add a bibliography block to the postdoc
    
    
    
    \end{document}
